\documentclass[manual-fr.tex]{subfiles}
\begin{document}
\SEM\ dispose de module indépendants les uns des autres, le programme principal faisant alors office d'aiguilleur vers le module à lancer.\\

Pour obtenir la liste des modules disponibles et la syntaxe générale pour les lancer :

python -m sem (--help ou -h)\\

Pour connaître la version de \SEM\ :

python -m sem (--version ou -v)\\

Pour connaître les informations relatives à la dernière version de \SEM\ :

python -m sem (--informations ou -i)\\

Pour lancer un module, la syntaxe générale est :

python -m sem $<$nom\_du\_module$>$ $<$arguments\_et\_options\_du\_module$>$\\

Les différents modules seront détaillés un par un.

\subsection{annotate}
\subfile{fr/subsection/module-annotate}

\subsection{chunking\_fscore}
\subfile{fr/subsection/module-chunking_fscore}

\subsection{clean}
\subfile{fr/subsection/module-clean_info}

\subsection{enrich}
\subfile{fr/subsection/module-enrich}

\subsection{export}
\subfile{fr/subsection/module-export}

\subsection{label\_consistency}
\subfile{fr/subsection/module-label_consistency}

\subsection{tagging}
\subfile{fr/subsection/module-tagging}

\subsection{segmentation}
\subfile{fr/subsection/module-segmentation}

\subsection{compile}
\subfile{fr/subsection/module-compile}

\subsection{decompile}
\subfile{fr/subsection/module-decompile}

\subsection{tagger}
\subfile{fr/subsection/module-tagger}

\subsection{gui}
\subfile{fr/subsection/module-gui}

\subsection{annotation\_gui}
\subfile{fr/subsection/module-annotation_gui}

\end{document}