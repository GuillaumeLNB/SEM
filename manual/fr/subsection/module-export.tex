\documentclass[manual-fr.tex]{subfiles}
\begin{document}

\begin{itemize}
    \item[] \textbf{description}
        \begin{itemize}
            \item[] Transforme des données du format CoNLL vers un autre format spécifié en argument.
        \end{itemize}
    \item[] \textbf{arguments}
        \begin{itemize}
            \item[] infile
                \begin{itemize}
                    \item[] le fichier d'entrée au format CoNLL.
                \end{itemize}
            \item[] exporter\_name
                \begin{itemize}
                    \item[] le nom du format d'export.
                \end{itemize}
            \item[] outfile
                \begin{itemize}
                    \item[] le fichier de sortie.
                \end{itemize}
        \end{itemize}
    \item[] \textbf{options}
        \begin{itemize}
            \item[] --help ou -h : switch
                \begin{itemize}
                    \item[] affiche l'aide
                \end{itemize}
            \item[] --pos-column ou -p
                \begin{itemize}
                    \item[] la colonne où l'information des parties du discours se trouvent.
                \end{itemize}
            \item[] --chunk-column ou -c
                \begin{itemize}
                    \item[] la colonne où l'information du chunking se trouvent.
                \end{itemize}
            \item[] --ner-column ou -n
                \begin{itemize}
                    \item[] la colonne où l'information de la reconnaissance des entités nommées se trouve.
                \end{itemize}
            \item[] --lang
                \begin{itemize}
                    \item[] la langue du document (défaut: fr)
                \end{itemize}
            \item[] --lang-style ou -s
                \begin{itemize}
                    \item[] la feuille de style CSS à utiliser pour l'export HTML (défaut: default.css)
                \end{itemize}
            \item[] --input-encoding : string
                \begin{itemize}
                    \item[] définit l'encodage du fichier d'entrée. Prioritaire sur la valeur de --encoding (défaut : --encoding).
                \end{itemize}
            \item[] --output-encoding : string
                \begin{itemize}
                    \item[] définit l'encodage du fichier de sortie. Prioritaire sur la valeur de --encoding (défaut : --encoding).
                \end{itemize}
            \item[] --encoding : string
                \begin{itemize}
                    \item[] définit l'encodage du fichier d'entrée et de sortie. Si un encodage est fourni pour un fichier,
                        cette valeur est surchargée (défaut : UTF-8).
                \end{itemize}
            \item[] --log ou -l : string
                \begin{itemize}
                    \item[] définit le niveau de log : info, warn ou critical (défaut : critical).
                \end{itemize}
            \item[] --log-file : fichier
                \begin{itemize}
                    \item[] le fichier où écrire le log (défaut : sortie terminal).
                \end{itemize}
        \end{itemize}
\end{itemize}

\end{document}