\documentclass[manual-fr.tex]{subfiles}
\begin{document}
Un fichier linéaire est un fichier dans lequel les mots sont (souvent) séparés par un espace. Ils représentent la majorité des textes (texte brut). \SEM\ considère qu'un retour à la ligne termine une phrase, lorsqu'il fournit en sortie un fichier linéaire, chaque phrase sera séparée par un retour à la ligne. Si un fichier en entrée est un fichier linéaire, \SEM\ pourra le segmenter en tokens et phrases.\\

\SEM\ ne peut traiter en entrée que les fichiers de texte brut.
\end{document}
