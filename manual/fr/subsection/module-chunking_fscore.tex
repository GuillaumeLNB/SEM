\documentclass[manual-fr.tex]{subfiles}
\begin{document}

\begin{itemize}
    \item[] \textbf{description}
        \begin{itemize}
            \item[] Calcule la f-mesure sur des données étiquetées selon un schéma BIO. Fournit une f-mesure par classe, une micro f-mesure globale et une macro f-mesure globale.
        \end{itemize}
    \item[] \textbf{arguments}
        \begin{itemize}
            \item[] infile
                \begin{itemize}
                    \item[] Le fichier contenant les données étiquetées à évaluer. Ce fichier est au format tabulaire type CoNLL 2003. Ce script est similaire à conlleval.
                \end{itemize}
        \end{itemize}
    \item[] \textbf{options}
        \begin{itemize}
            \item[] --help ou -h : switch
                \begin{itemize}
                    \item[] affiche l'aide
                \end{itemize}
            \item[] --reference-column ou -r : int
                \begin{itemize}
                    \item[] l'indice de la colonne où se trouvent les étiquettes de références (défaut : -2).
                \end{itemize}
            \item[] --tagging-column ou -t : int
                \begin{itemize}
                    \item[] l'indice de la colonne où se trouvent les étiquettes hypothèses fournies par le système (défaut : -1).
                \end{itemize}
            \item[] --input-encoding : string
                \begin{itemize}
                    \item[] définit l'encodage du fichier d'entrée. Prioritaire sur la valeur de --encoding (défaut : --encoding).
                \end{itemize}
            \item[] --output-encoding : string
                \begin{itemize}
                    \item[] définit l'encodage du fichier de sortie. Prioritaire sur la valeur de --encoding (défaut : --encoding).
                \end{itemize}
            \item[] --encoding : string
                \begin{itemize}
                    \item[] définit l'encodage du fichier d'entrée et de sortie. Si un encodage est fourni pour un fichier,
                        cette valeur est surchargée (défaut : UTF-8).
                \end{itemize}
            \item[] --log ou -l : string
                \begin{itemize}
                    \item[] définit le niveau de log : info, warn ou critical (défaut : critical).
                \end{itemize}
            \item[] --log-file : fichier
                \begin{itemize}
                    \item[] le fichier où écrire le log (défaut : sortie terminal).
                \end{itemize}
        \end{itemize}
\end{itemize}

\end{document}