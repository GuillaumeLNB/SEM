\documentclass[manual-fr.tex]{subfiles}
\begin{document}

\begin{itemize}
    \item[] \textbf{description}
        \begin{itemize}
            \item[] Améliore la cohérence des annotations en diffusant dans le document les annotations faites par le système. Les éléments non-annotés identiques à des éléments annotés seront annotés selon la catégorie la plus fréquente.
        \end{itemize}
    \item[] \textbf{arguments}
        \begin{itemize}
            \item[] infile
                \begin{itemize}
                    \item[] le fichier d'entrée (format CoNLL).
                \end{itemize}
        \end{itemize}
        \begin{itemize}
            \item[] outfile
                \begin{itemize}
                    \item[] le fichier de sortie (format CoNLL).
                \end{itemize}
        \end{itemize}
    \item[] \textbf{options}
        \begin{itemize}
            \item[] --help ou -h : switch
                \begin{itemize}
                    \item[] affiche l'aide
                \end{itemize}
            \item[] --token-column ou -t
                \begin{itemize}
                    \item[] la colonne où l'information des tokens se trouve.
                \end{itemize}
            \item[] --tag-column ou -c
                \begin{itemize}
                    \item[] la colonne où l'information des étiquettes se trouve.
                \end{itemize}
            \item[] --label-consistency (choix: non-overriding, overriding)
                \begin{itemize}
                    \item[] l'heuristique de diffusion. "non-overriding" laisse les annotations du systèmes telles quelles. "overriding" écrase les annotations du système si une annotation plus longue a est trouvée.
                \end{itemize}
            \item[] --input-encoding : string
                \begin{itemize}
                    \item[] définit l'encodage du fichier d'entrée. Prioritaire sur la valeur de --encoding (défaut : --encoding).
                \end{itemize}
            \item[] --output-encoding : string
                \begin{itemize}
                    \item[] définit l'encodage du fichier de sortie. Prioritaire sur la valeur de --encoding (défaut : --encoding).
                \end{itemize}
            \item[] --encoding : string
                \begin{itemize}
                    \item[] définit l'encodage du fichier d'entrée et de sortie. Si un encodage est fourni pour un fichier,
                        cette valeur est surchargée (défaut : UTF-8).
                \end{itemize}
            \item[] --log ou -l : string
                \begin{itemize}
                    \item[] définit le niveau de log : info, warn ou critical (défaut : critical).
                \end{itemize}
            \item[] --log-file : fichier
                \begin{itemize}
                    \item[] le fichier où écrire le log (défaut : sortie terminal).
                \end{itemize}
        \end{itemize}
\end{itemize}

\end{document}