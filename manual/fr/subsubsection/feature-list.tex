\documentclass[manual-fr.tex]{subfiles}
\begin{document}

La feature \textit{list} est une feature booléenne qui permet de définir une liste non bornée de propriétés booléennes qui seront évaluées. Une feature de type \textit{list} dispose des actions suivantes : \textit{none} (toutes les features doivent être évaluées à faux), \textit{some} (au moins une feature doit être évaluée à vrai) et \textit{all} (toutes les features doivent être évaluées à vrai).

\begin{figure}[ht!]
\footnotesize
\begin{xml}
\xmarker{list}{ \xfield{name}{top-of-hierarchy} \xfield{action}{some}}{\\
    \xmarker{string}{ \xfield{action}{equal}}{PDG}\\
    \xmarker{regexp}{ \xfield{action}{check} \xfield{entry}{lower}}{\^{}(président|directeur)\$}\\
}
\end{xml}
\caption{exemple de la feature \textit{list} "some".}
\label{fig:feature-list-some}
\end{figure}
\end{document}