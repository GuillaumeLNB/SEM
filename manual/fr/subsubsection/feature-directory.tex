\documentclass[manual-fr.tex]{subfiles}
\begin{document}

Les \textit{features} directory permettent d'utiliser des répertoires de lexiques comme définis dans \cite{dupont2017exploration}. Deux features sont définies :
\begin{itemize}
    \item directory (\textit{sequence feature}) : applique un répertoire de lexiques. Les tokens non-reconnus sont remplacés par "O". La \textit{feature} attend un champ "path" contenant le chemin vers le dossier contenant l'ensemble des lexique
    \item fill (\textit{string feature}) : remplace l'élément par le contenu de l'entrée donnée dans le champ "filler-entry" si ce dernier est reconnu par la \textit{boolean feature} donnée en argument.
\end{itemize}

\begin{figure}[ht!]
\footnotesize
\begin{xml}
\xunit{directory}{\xfield{name}{NER-ontology} \xfield{path}{../../dictionaries/fr/NER-ontology}}
\end{xml}
\caption{exemple de la feature \textit{directory}.}
\label{fig:feature-directory}
\end{figure}

\begin{figure}[ht!]
\footnotesize
\begin{xml}
\xmarker{fill}{ \xfield{name}{NER-ontology-POS} \xfield{entry}{NER-ontology} \xfield{filler-entry}{POS}}{\\
    \xmarker{string}{ \xfield{action}{equal}}{O}\\
}
\end{xml}
\caption{exemple de la feature \textit{directory} "fill".}
\label{fig:feature-directory-fill}
\end{figure}
\end{document}