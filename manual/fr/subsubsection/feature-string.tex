\documentclass[manual-fr.tex]{subfiles}
\begin{document}

Les \textit{token features} string définissent des opérations de base sur les chaines de caractères. Les actions suivantes sont définies :
\begin{itemize}
    \item equal (\textit{boolean feature}) : vérifie l'égalité de la chaine en entrée par rapport à la chaine en argument. Définit les options suivantes :
    \begin{itemize}
        \item casing="(sensitive|s|insensitive|i)" : définit la sensibilité à la casse de la comparaison (défaut : "sensitive").
    \end{itemize}
\end{itemize}

\begin{figure}[ht!]
\footnotesize
\begin{xml}
\xmarker{string}{ \xfield{action}{equal} \xfield{casing}{sensitive}}{O}
\end{xml}
\caption{exemple de la feature \textit{string} "equal".}
\label{fig:feature-string-equal}
\end{figure}

\end{document}