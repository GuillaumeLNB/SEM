\documentclass[manual-fr.tex]{subfiles}
\begin{document}

La \textit{sequence feature} rule permet d'intégrer des règles en tant que features. Les arguments d'une feature rule sont systématiquement des \textit{boolean features} ont tous un champ "card" qui indique la cardinalité de la feature. Les différentes valeurs pour "card" sont :
\begin{itemize}
    \item ? : 0 ou 1 fois
    \item * : 0 ou un nombre illimité de fois
    \item + : 1 ou un nombre illimité de fois
    \item [entier] : exactement [entier] fois
    \item "min,max" : au moins min fois et au maximum max fois.
\end{itemize}

Un argument spécifique des features rule est "orrule" qui permet de reconnaitre une règle parmi plusieurs au choix. Un exemple de feature rule est illustré dans la figure \ref{fig:feature-rule}.

\begin{figure}[ht!]
\footnotesize
\begin{xml}
\xmarker{rule}{ \xfield{name}{amount}}{\\
    \xmarker{list}{ \xfield{action}{some}}{\\
        \xmarker{regexp}{ \xfield{action}{check} \xfield{entry}{word}}{\^{}[0-9]+\$}\\
        \xmarker{regexp}{ \xfield{action}{check} \xfield{entry}{lower}}{\^{}(une?|deux|trois|quatre|cinq|six|sept|huit|neuf|dix
|onze|douze|treize|quatorze|quinze|seize|dix-sept|dix-huit|dix-neuf|vingt|trente|quarante|cinquante|soixante
|soixante-dix|quatre-vingt|quatre-vingt-dix|cents?|mille|millions?|milliards?)\$}\\
    }\\
    \xmarker{orrule}{}{\\
        \xmarker{regexp}{ \xfield{action}{check} \xfield{entry}{chunking} \xfield{card}{+}}{-NP\$}\\
        \xmarker{regexp}{ \xfield{action}{check} \xfield{entry}{chunking} \xfield{card}{+}}{-PP\$}\\
    }\\
}
\end{xml}
\caption{exemple de la feature \textit{rule}.}
\label{fig:feature-rule}
\end{figure}

\end{document}