\documentclass[manual-fr.tex]{subfiles}
\begin{document}

Les \textit{features} dictionary définissent des features se basant sur des lexiques. Les actions suivantes sont possibles :
\begin{itemize}
    \item token (\textit{boolean feature}) : vérifie l'appartenance d'un token au lexique ;
    \item multiword (\textit{sequence feature}) : cherche les séquences de tokens appartenant au lexique.
\end{itemize}

Des exemples des ces features sont illustrées dans les figures \ref{fig:feature-dictionary-token} et \ref{fig:feature-dictionary-multiword}.

\begin{figure}[ht!]
\footnotesize
\begin{xml}
\xunit{dictionary}{\xfield{name}{token-dictionary} \xfield{action}{token} \xfield{path}{path/to/token-dictionary}}
\end{xml}
\caption{exemple de la feature \textit{dictionary} "token".}
\label{fig:feature-dictionary-token}
\end{figure}

\begin{figure}[ht!]
\footnotesize
\begin{xml}
\xunit{dictionary}{\xfield{name}{multiword-dictionary} \xfield{action}{multiword} \xfield{path}{path/to/multiword-dictionary}}
\end{xml}
\caption{exemple de la feature \textit{dictionary} "multiword".}
\label{fig:feature-dictionary-multiword}
\end{figure}

\end{document}