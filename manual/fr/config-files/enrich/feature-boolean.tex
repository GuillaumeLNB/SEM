\documentclass[manual-fr.tex]{subfiles}
\begin{document}

Les \textit{boolean features} \textit{boolean} définissent des expressions booléennes. Un exemple de feature \textit{boolean} est donné dans la figure \ref{fig:feature-boolean}. Trois actions sont disponibles :
\begin{itemize}
    \item and : "et" logique. Prend deux \textit{boolean features} en argument.
    \item or : "ou" logique. Prend deux \textit{boolean features} en argument.
    \item not : "non" logique. Prend une \textit{boolean feature} en argument.
\end{itemize}

\begin{figure}[ht!]
\footnotesize
\begin{xml}
\xmarker{boolean}{ \xfield{name}{StartsWithUpper-AndNot-FirstWordOfSentence} \xfield{action}{and}}{\\
    \xmarker{unary}{ \xfield{action}{isUpper}}{0}\\
    \xmarker{boolean}{ \xfield{action}{not}}{\\
        \xunit{nullary}{\xfield{action}{BOS}}\\
    }\\
}
\end{xml}
\caption{exemple de la feature \textit{boolean}.}
\label{fig:feature-boolean}
\end{figure}

\end{document}