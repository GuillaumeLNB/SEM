\documentclass[manual-fr.tex]{subfiles}
\begin{document}

Le fichier de configuration du module tagger est appelé le fichier de configuration maître. Il permet de définir une séquence de traitements
(modules) ainsi que des options globales aux différents modules qui seront lancés les uns après les autres.\\

Le fichier maître est un fichier XML de type de document ``master''. Il a deux parties : une ``pipeline'' qui est une séquences de modules et
une ``options'' qui permet de définir les options globales. Un exemple de pipeline est illustré dans la figure \ref{fig:sem-pipeline}.\\

\begin{figure}[ht!]
\footnotesize
\begin{xml}
\xheader{xml version="1.0" encoding="UTF-8"}\\
\xmarker{master}{}{\\
  \xmarker{pipeline}{}{\\
    \xunit{segmentation}{\xfield{tokeniser}{fr}}\\
    \xunit{enrich}{\xfield{informations}{pos.xml} \xfield{mode}{label}}\\
    \xunit{annotate}{\xfield{model}{models/POS} \xfield{field}{POS}}\\
    \xunit{clean}{\xfield{to-keep}{word,POS}}\\
    \xunit{enrich}{\xfield{informations}{NER.xml} \xfield{mode}{label}}\\
    \xunit{annotate}{\xfield{model}{models/NER} \xfield{field}{NER}}\\
    \xunit{clean}{\xfield{to-keep}{word,POS,NER}}\\
  }\\
\\
  \xmarker{options}{}{\\
    \xunit{encoding}{\xfield{input-encoding}{utf-8} \xfield{output-encoding}{utf-8}}\\
    \xunit{log}{\xfield{log\_level}{INFO}}\\
    \xunit{export}{\xfield{format}{html} \xfield{pos}{POS} \xfield{ner}{NER} \xfield{lang}{fr} \xfield{lang\_style}{default.css}}\\
  }\\
}
\end{xml}
\caption{Spécification d'un pipeline de SEM. Les pipelines permettent de définir une séquence de traitements ainsi que certaines options globales.}
\label{fig:sem-pipeline}
\end{figure}

\end{document}