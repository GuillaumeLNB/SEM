\documentclass[12pt]{article}

\usepackage[utf8]{inputenc}
\usepackage[T1]{fontenc}
\usepackage[french]{babel}

\usepackage[usenames,dvipsnames]{color}

\usepackage[colorlinks=true, citecolor=blue, linkcolor=ForestGreen]{hyperref}

\usepackage{macros}
\usepackage[many]{tcolorbox}

\usepackage[top=1in, bottom=1.5in, left=1in, right=1in]{geometry}

\usepackage{listings}
\lstdefinestyle{pieceofcode}{ 
  basicstyle=\bfseries\color{white},
  backgroundcolor=\color{black},
  stepnumber=2,
  identifierstyle=\color{white},
  stringstyle=\color{white},
  keywordstyle=\color{white},
  commentstyle=\color{white},
}

\usepackage{subfiles}

\title{\SEMFull}
\date{}

\begin{document}
    \maketitle
    
    \tableofcontents
    \newpage
    
    \section{Préface}
    
        \subsection{Présentation de \SEM}
        \subfile{fr/subsection/presentation}
    
    \section{Installation}
    \subfile{fr/section/installation}

        \subsection{Si GIT est installé}
        \subfile{fr/subsection/git-installed}

        \subsection{Si GIT n'est pas installé}
        \subfile{fr/subsection/git-not-installed}

        \subsection{\Wapiti}
        \subfile{fr/subsection/wapiti}
    
    \section{Corpus, annotations et ressources linguistiques}

        \subsection{\FTBFull}
        \subfile{fr/subsection/ftb}

        \subsection{Jeu d'annotation PoS}
        \subfile{fr/subsection/tags-pos}

        \subsection{Annotation en chunks}
        \subfile{fr/subsection/tags-chunking}

        \subsection{Annotation en entités nommées}
        \subfile{fr/subsection/tags-ner}

        \subsection{\LeFFFFull}
        \subfile{fr/subsection/lefff}
    
    \section{Formats des fichiers}
    \subfile{fr/section/file_formats}

        \subsection{fichiers linéaires}
        \subfile{fr/subsection/file-linear}

            \subsubsection{Exemples}
            \subfile{fr/subsubsection/file-linear-examples}

        \subsection{fichiers vectorisés}
        \subfile{fr/subsection/file-vectorised}

            \subsubsection{Exemples}
            \subfile{fr/subsubsection/file-vectorised-examples}

        \subsection{fichiers SEM}
        \subfile{fr/subsection/file-sem}

            \subsubsection{Exemples}
            \subfile{fr/subsubsection/file-sem-examples}
    
    \section{Utilisation}
    \subfile{fr/section/usage}

        \subsection{annotate}
        \subfile{fr/subsection/module-annotate}

        \subsection{chunking\_fscore}
        \subfile{fr/subsection/module-chunking_fscore}

        \subsection{clean}
        \subfile{fr/subsection/module-clean_info}

        \subsection{enrich}
        \subfile{fr/subsection/module-enrich}

        \subsection{export}
        \subfile{fr/subsection/module-export}

        \subsection{label\_consistency}
        \subfile{fr/subsection/module-label_consistency}

        \subsection{tagging}
        \subfile{fr/subsection/module-tagging}

        \subsection{segmentation}
        \subfile{fr/subsection/module-segmentation}

        \subsection{compile}
        \subfile{fr/subsection/module-compile}

        \subsection{decompile}
        \subfile{fr/subsection/module-decompile}

        \subsection{tagger}
        \subfile{fr/subsection/module-tagger}

        \subsection{gui}
        \subfile{fr/subsection/module-gui}

        \subsection{annotation\_gui}
        \subfile{fr/subsection/module-annotation_gui}
    
    \section{Fichiers de configuration}

        \subsection{Pour le module enrich}
        \subfile{fr/subsection/config-enrich}

            \subsubsection{Les features \textit{arity}}
            \subfile{fr/subsubsection/feature-arity}

            \subsubsection{Les features \textit{boolean}}
            \subfile{fr/subsubsection/feature-boolean}

            \subsubsection{Les features \textit{dictionary}}
            \subfile{fr/subsubsection/feature-dictionary}

            \subsubsection{Les features \textit{directory}}
            \subfile{fr/subsubsection/feature-directory}

            \subsubsection{Les features \textit{list}}
            \subfile{fr/subsubsection/feature-list}

            \subsubsection{Les features \textit{matcher}}
            \subfile{fr/subsubsection/feature-matcher}

            \subsubsection{Les features \textit{rule}}
            \subfile{fr/subsubsection/feature-rule}

            \subsubsection{Les features \textit{string}}
            \subfile{fr/subsubsection/feature-string}

            \subsubsection{Les features \textit{trigger}}
            \subfile{fr/subsubsection/feature-triggered}

        \subsection{Pour le module tagger}
        \subfile{fr/subsection/config-tagger}
    
    \section{Réentraîner \SEM}
    \subfile{fr/section/retrain-sem}
    
        \subsection{Réentraîner SEM depuis des fichiers déjà annotés}
        
            \subsubsection{Lancer la GUI de SEM}
            \subfile{fr/subsubsection/launch-sem-gui}
            
            \subsubsection{Sélectionner les données et leur prétraitement}
            \subfile{fr/subsubsection/data-and-preprocess}
            
            \subsubsection{Lancer l'entraînement}
            \label{subsubsec:train-SEM-from-annotated}
            \subfile{fr/subsubsection/train-SEM-from-annotated}
        
        \subsection{Réentraîner SEM depuis des fichiers non annotés}
        \subfile{fr/subsection/retrain-sem-from-unannotated}
        
            \subsubsection{Lancer la GUI de SEM pour l'annotation manuelle}
            \subfile{fr/subsubsection/launch-sem-annotation-gui}
        
            \subsubsection{Annoter manuellement avec la GUI de SEM}
            \subfile{fr/subsubsection/manually-annotate-with-sem}
        
        \subsection{Utiliser le nouveau modèle}
        \subfile{fr/subsection/use-new-model}
    
    \bibliographystyle{apa}
    \bibliography{biblio}
\end{document}
